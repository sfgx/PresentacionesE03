\begin{frame}{}
	

\end{frame}


%
\begin{varblock}{}{fg = hi_blue}
	  
\end{varblock}

%
\begin{varblock}{}{fg = vi_magenta}
	  
\end{varblock}

%
\begin{alertblobk}{}

\end{alertblock}

%
\begin{exampleblobk}{}

\end{exampleblock}

\begin{enumerate}[label=(\alph*)]
    \item 
\end{enumerate}






%%
% Para incorporar más adelante
%%
\setbeamercolor{progress bar}{fg=rb_green, bg=rb_green}
\section{Funciones Cuasicóncavas y Cuasiconvexas}
\setbeamercolor{frametitle}{bg=rb_green, fg = white}

\begin{frame}{Función cuasicóncava y estrictamente cóncava}

	\begin{varblock}{Función cuasicóncava}{fg = hi_blue}
		$f: D \rightarrow \mathbb{R}$ es una función cuasicóncava si para todo $\mathbf{x}^1, \mathbf{x}^2 \in D$,
		
		\[ f(\mathbf{x}^t) \geq  \min [f(\mathbf{x}^1), f(\mathbf{x}^2)]   \quad   \forall \, t \in [0,1] \]
	\end{varblock}

	\begin{varblock}{Función estrictamente cuasicóncava}{fg = hi_blue}
		$f: D \rightarrow \mathbb{R}$ es una función estrictamente cuasicóncava si para todo $\mathbf{x}^1 \neq  \mathbf{x}^2 \in D$,
		
		\[ f(\mathbf{x}^t) > \min [f(\mathbf{x}^1), f(\mathbf{x}^2)]   \quad   \forall \, t \in (0,1) \]
	\end{varblock}

\end{frame}

\begin{frame}{Cuasiconcavidad y Conjuntos Superiores}

	\begin{varblock}{Teorema}{fg = hi_blue}
		$f: D \rightarrow \mathbb{R}$ es una función cuasicóncava si y solo si $S(y)$ es un conjunto convexo
		para todo $y \in \mathbb{R}$,
	\end{varblock}

\end{frame}


